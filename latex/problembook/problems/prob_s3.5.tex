%
% Copyright 2018 Joel Feldman, Andrew Rechnitzer and Elyse Yeager.
% This work is licensed under a Creative Commons Attribution-NonCommercial-ShareAlike 4.0 International License.
% https://creativecommons.org/licenses/by-nc-sa/4.0/
%
\questionheader{ex:s3.5}

%%%%%%%%%%%%%%%%%%
\subsection*{\Conceptual}
%%%%%%%%%%%%%%%%%%

\begin{Mquestion}
	Suppose $f(x)=\displaystyle\sum_{n=0}^\infty\left(\dfrac{3-x}{4}\right)^n$. What is $f(1)$?
\end{Mquestion}
\begin{hint}
	$f(1)$ is the sum of a geometric series.
\end{hint}
\begin{answer}
2
\end{answer}
\begin{solution}
	\begin{align*}
f(1)&=\displaystyle\sum_{n=0}^\infty\left(\dfrac{3-1}{4}\right)^n\\
&=\displaystyle\sum_{n=0}^\infty\left(\frac12\right)^n
\intertext{This is a geometric series with $r=\frac{1}{2}$, so we know that it converges and}
&=\frac{1}{1-\frac12}=2
\end{align*}

The question does not ask us to find the interval of convergence of the series defining $f(x)$.
But we will do so anyway, to get a bit more practice. We may rewrite the $n^{\rm th}$ term of the series defining $f(x)$ as
\begin{align*}
\left(\frac{3-x}{4}\right)^n = a r^n\qquad\text{with $a=1$ and $r= \frac{3-x}{4}$}
\end{align*}
That is, for every fixed $x$, we have a geometric series with  $r= \frac{3-x}{4}$.
So, by (\eref{CLP101}{eq:geomsum}) in the CLP-2 text, the series converges if and only if
\begin{align*}
&|r|= \left|\frac{3-x}{4}\right|<1 \\
\iff & -1< \frac{3-x}{4} < 1 \\
\iff & -4< 3-x < 4 \\
\iff & -4< 3-x \ \text{and}\ 3-x < 4  \\
\iff & -1< x < 7
\end{align*}

\end{solution}

\begin{Mquestion}
Suppose $f(x)=\displaystyle\sum_{n=1}^\infty \dfrac{(x-5)^n}{n!+2}$. Give a power series representation of $f'(x)$.
\end{Mquestion}
\begin{hint}
	Calculate $\displaystyle\diff{}{x}\left\{\frac{(x-5)^n}{n!+2}\right\}$ when $n$ is a constant.
\end{hint}
\begin{answer}
	$f(x)=\displaystyle\sum_{n=1}^\infty \dfrac{n(x-5)^{n-1}}{n!+2}$
\end{answer}
\begin{solution}
	By Theorem~\eref{CLP101}{thm:SRpsops} in the CLP-2 text, we may differentiate our function term-by-term for all $x$ obeying $|x-5|<R$, where $R$ is the radius of convergence of the power series. The series defining $f(x)$ is reminiscent of the exponential series 
$\sum_{n=0}^\infty \frac{X^n}{n!}$ of Example \eref{CLP101}{eg:PWRb} in the CLP-2 text.
In that example, we showed that $\sum_{n=0}^\infty \frac{X^n}{n!}$ has radius of convergence $\infty$. Since
\begin{align*}
\left|\frac{(x-5)^n}{n!+2}\right|\le \frac{X^n}{n!}\quad\text{with}\  X=|x-5|
\end{align*}
the comparison test, Theorem \eref{CLP101}{thm:SRcomparisonTest} in the CLP-2 text, tells us
that $\sum_{n=1}^\infty \frac{(x-5)^n}{n!+2}$ converges for all $x$. So we may 
differentiate our function term-by-term.
	\begin{align*}
	f(x)&=\displaystyle\sum_{n=1}^\infty \dfrac{(x-5)^n}{n!+2}\\
	f'(x)&=\displaystyle\sum_{n=1}^\infty \diff{}{x}\left\{\dfrac{(x-5)^n}{n!+2}\right\}\\
	&=\sum_{n=1}^\infty \frac{n(x-5)^{n-1}}{n!+2}
		\end{align*}
	Keep in mind that $x$ is our variable, and \emph{for each term}, $n$ is constant.
\end{solution}

\begin{question}
	Let $f(x)=\displaystyle \sum_{n=a}^\infty A_n(x-c)^n$ for some positive constants $a$ and $c$, and some sequence of constants $\{A_n\}$. For which values of $x$ does $f(x)$ definitely converge?
\end{question}
\begin{hint}
	There is only one.
\end{hint}
\begin{answer}
	only $x=c$
\end{answer}
\begin{solution}
	If $x=c$, then
	\begin{align*}
	f(x)&=A_a(c-c)^a+A_{a+1}(c-c)^{a+1}+A_{a+2}(c-c)^{a+2}+\cdots\\
	&=A_a \cdot 0+A_{a+1}\cdot 0+A_{a+2}\cdot 0+\cdots\\
	&=0
	\end{align*}
	 So, $f(x)$ converges (to the constant 0) when $x=c$. (Had we allowed $a=0$, it would be possible for $f(x)$ to converge to a nonzero number $A_0$, because we use the convention $0^0=1$.)

	 Depending on the sequence $\{A_n\}$,  it's possible that $f(x)$ diverges for all $x \neq c$. For example, suppose $A_n=n!$, so $f(x)=\displaystyle\sum_{n=0}^\infty n!(x-c)^n$. If $x \neq c$, then the limit $\displaystyle\lim_{n\to\infty}	 \left|\frac{(n+1)!(x-c)^{n+1}}{n!(x-c)^n}\right|=\lim_{n\to\infty}(n+1)|x-c|$  is infinity, since $x-c\neq 0$. So, the series diverges.

	 We've now shown that the series definitely converges at $x=c$, but at any other point, it may fail to converge.
\end{solution}



\begin{Mquestion}
	Let $f(x)$ be a power series centred at $c=5$. If $f(x)$ converges at $x=-1$, and diverges at $x=11$, what is the radius of convergence of $f(x)$?
\end{Mquestion}
\begin{hint}
	Use Theorem~\eref{CLP101}{thm:SRradiusofconvergence} in the CLP-2 text.
\end{hint}
\begin{answer}
	$R=6$
\end{answer}
\begin{solution}
	According to Theorem~\eref{CLP101}{thm:SRradiusofconvergence} in the CLP-2 text, because $f(x)$ diverges somewhere, and because it converges at a point other than its centre,  $f(x)$ has a positive radius of convergence $R$. That is, $f(x)$ converges whenever $|x-5|<R$, and it diverges whenever $|x-5|>R$.

	Since we are told that the series \emph{diverges} at $x=11$, the statement $|11-5|<R$ must be false. That is, we must have $R\le |11-5|=6$.

	Since we are told that the series \emph{converges} at $x=-1$, the statement $|-1-5|>R$ must be false. That is, we must have $R\ge |-1-5|=6$.
	
	Therefore, $R=6$.
\end{solution}









%%%%%%%%%%%%%%%%%%
\subsection*{\Procedural}
%%%%%%%%%%%%%%%%%%

\begin{Mquestion}[M105 2012A]
(a) Find the radius of convergence of the series
\begin{equation*}
\sum_{k=0}^\infty (-1)^k 2^{k+1} x^k
\end{equation*}

\noindent (b)
You are given the formula for the sum of a geometric series, namely:
\begin{equation*}
1+r+r^2 + \cdots =\frac{1}{1-r},\qquad|r|<1
\end{equation*}
Use this fact to evaluate the series in part (a).
\end{Mquestion}

\begin{hint}
Review the discussion immediately following Definition
\eref{CLP101}{def:SRpowerSeries} in the
%\href{http://www.math.ubc.ca/%7Efeldman/m101/clp/clp_notes_101.pdf}{CLP-2 text}.
CLP-2 text.
\end{hint}

\begin{answer}
(a) $R = \dfrac{1}{2}$\qquad
(b) $\dfrac{2}{1+2x}$ for all $|x|<\dfrac{1}{2}$
\end{answer}

\begin{solution} (a)
We apply the ratio test for the series whose $k^{\rm th}$
term is $a_k= (-1)^k 2^{k+1} x^k$. Then
\begin{align*}
\lim_{k\to\infty} \bigg| \frac{ a_{k+1} }{ a_k } \bigg|
&= \lim_{k\to\infty} \bigg| \frac{(-1)^{k+1} 2^{k+2} x^{k+1}}
                            {(-1)^k 2^{k+1} x^k} \bigg| \\
&= \lim_{k\to\infty}|2x| = |2x|
\end{align*}
Therefore, by the ratio test, the series converges for all $x$
obeying $|2x|<1$, i.e. $|x|<\frac{1}{2}$, and diverges for all $x$
obeying $|2x|>1$, i.e. $|x|>\frac{1}{2}$.
So the radius of convergence is $R = \frac{1}{2}$.

Alternatively, one can set $A_k = (-1)^k 2^{k+1}$ and compute
\begin{equation*}
A = \lim_{k\to\infty} \bigg| \frac{A_{k+1}}{A_k}\bigg|
  = \lim_{k\to\infty} \bigg| \frac{(-1)^{k+1} 2^{k+2}}{(-1)^k 2^{k+1}}\bigg|
  = \lim_{k\to\infty} 2
  =2
\end{equation*}
so that $R=\frac{1}{A}=\frac{1}{2}$, again.

\noindent (b) The series is
\begin{align*}
\sum_{k=0}^\infty (-1)^k 2^{k+1} x^k
=2 \sum_{k=0}^\infty (-2x)^k
=2\sum_{k=0}^\infty r^k\Big|_{r=-2x}
=2\times\frac{1}{1-r}
=\frac{2}{1+2x}
\end{align*}
for all $|r|=|2x|<1$, i.e. all $|x|<\frac{1}{2}$.


\end{solution}
%%%%%%%%%%%%%%%%%%%



\begin{Mquestion}[M105 2014A]
Find the radius of convergence for the power series
$\displaystyle\sum_{k=0}^\infty \frac{x^k}{10^{k+1}(k+1)!}$
\end{Mquestion}

\begin{hint}
Review the discussion immediately following Definition
\eref{CLP101}{def:SRpowerSeries} in the
%\href{http://www.math.ubc.ca/%7Efeldman/m101/clp/clp_notes_101.pdf}{CLP-2 text}.
CLP-2 text.
\end{hint}

\begin{answer}
$R = \infty$
\end{answer}

\begin{solution}
We apply the ratio test for the series whose $k^{\rm th}$
term is $a_k= \frac{x^k}{ 10^{k+1}(k+1)! }$. Then
\begin{align*}
\lim_{k\to\infty} \bigg| \frac{ a_{k+1} }{ a_k } \bigg|
&= \lim_{k\to\infty} \bigg| \frac{x^{k+1}}{10^{k+2}(k+2)!} \cdot
                             \frac{10^{k+1}(k+1)!}{x^k} \bigg| \\
&=\lim_{k\to\infty} \left|\frac{10^{k+1}}{10^{k+2}}\right|\cdot\left|\frac{(k+1)!}{(k+2)!}\right|\cdot\left|\frac{x^{k+1}}{x^k}\right|\\
&= \lim_{k\to\infty}\frac{1}{10(k+2)}|x| = 0 < 1
\end{align*}
for all $x$.
Therefore, by the ratio test, the series converges for all $x$
and the radius of convergence is $R = \infty$.

Alternatively, one can set $A_k = \dfrac{1}{10^{k+1}(k+1)!}$ and compute
$A = \displaystyle\lim_{k\to\infty} \left| \frac{A_{k+1}}{A_k}\right| = 0$, so that
$R$ is again $+\infty$.


\end{solution}
%%%%%%%%%%%%%%%%%%%


\begin{question}[2014A]
Find the radius of convergence for the power series
$\displaystyle\sum_{n=0}^\infty \frac{(x - 2)^n}{n^2+1}$\ .
\end{question}

\begin{hint}
Review the discussion immediately following Definition
\eref{CLP101}{def:SRpowerSeries} in the
%\href{http://www.math.ubc.ca/%7Efeldman/m101/clp/clp_notes_101.pdf}{CLP-2 text}.
CLP-2 text.
\end{hint}

\begin{answer}
$1$
\end{answer}

\begin{solution}
We apply the ratio test with $a_n = \frac{(x - 2)^n}{n^2+1}$.

\begin{align*}
\lim_{n\rightarrow\infty}\Big|\frac{a_{n+1}}{a_n}\Big|
&=\lim_{n\rightarrow\infty}\bigg|\frac{(x-2)^{n+1}}{(n+1)^2+1}\cdot
                               \frac{n^2+1}{(x-2)^n}\bigg|\\
&=\lim_{n\rightarrow\infty}\frac{n^2+1}{(n+1)^2+1}
                               |x-2|\\
&=\lim_{n\rightarrow\infty}\frac{1+1/n^2}{(1+1/n)^2+1/n^2}
                               |x-2|\\
&=|x-2|
\end{align*}
So, the series converges if $|x-2|<1$ and diverges if $|x-2|>1$.
That is, the radius of convergence is 1.
\end{solution}
%%%%%%%%%%%%%%%%%%%


\begin{question}[2013A]
 Consider the power series
 $\displaystyle\sum\limits_{n=1}^\infty \frac{(-1)^n(x+2)^n}{\sqrt{n}}$,
where $x$ is a real number. Find the interval of
convergence of this series.
\end{question}

\begin{hint}
See Example \eref{CLP101}{eg:SRintervalofconvergence} in the
%\href{http://www.math.ubc.ca/%7Efeldman/m101/clp/clp_notes_101.pdf}{CLP-2 text}.
CLP-2 text.
\end{hint}

\begin{answer}
The interval of convergence
is $-1<x+2\le 1$ or $(-3,-1]$.
\end{answer}

\begin{solution}
We apply the ratio test for the series whose $n^{\rm th}$
term is $a_n=\dfrac{(-1)^n(x+2)^n}{\sqrt{n}}$. Then
\begin{align*}
\lim_{n\rightarrow\infty}\bigg|\frac{a_{n+1}}{a_n}\bigg|
&=\lim_{n\rightarrow\infty} \bigg|\frac{(x+2)^{n+1}}{\sqrt{n+1}}
                                  \ \frac{\sqrt{n}}{(x+2)^n}\bigg| \\
&=\lim_{n\rightarrow\infty} |x+2|\ \frac{\sqrt{n}}{\sqrt{n+1}} \\
&=\lim_{n\rightarrow\infty} |x+2|\ \frac{1}{\sqrt{1+1/n}} \\
&=|x+2|
\end{align*}
So the series must converge when $|x+2|<1$ and must diverge when $|x+2|>1$.
When $x+2=1$, the series reduces to
\begin{align*}
\sum_{n=1}^\infty \frac{(-1)^n}{\sqrt{n}}
\end{align*}
which converges by the alternating series test.
When $x+2=-1$, the series reduces to
\begin{align*}
\sum_{n=1}^\infty \frac{1}{\sqrt{n}}
\end{align*}
which diverges by the $p$--series test with $p=\frac{1}{2}$.
So the interval of convergence is $-1<x+2\le 1$ or $(-3,-1]$.

\end{solution}
%%%%%%%%%%%%%%%%%%%

\begin{Mquestion}[2016Q5]
Find the radius of convergence and interval of convergence of the series
\begin{align*}
\sum_{n=0}^{\infty} \frac{(-1)^n}{n+1} \left(\frac{x+1}{3}\right)^n
\end{align*}
\end{Mquestion}

\begin{hint}
See Example \eref{CLP101}{eg:SRintervalofconvergence} in the
%\href{http://www.math.ubc.ca/%7Efeldman/m101/clp/clp_notes_101.pdf}{CLP-2 text}.
CLP-2 text.
\end{hint}

\begin{answer}
The interval of convergence is $-4<x\le2$, or simply $(-4,2]$.
\end{answer}

\begin{solution}
We apply the ratio test for the series whose $n^{\rm th}$
term is $a_n= \frac{(-1)^n}{n+1} \left(\frac{x+1}{3}\right)^n$.
\begin{align*}
\lim_{n\to\infty} \bigg| \frac{a_{n+1}}{a_n} \bigg|
&= \lim_{n\to\infty} \bigg| \frac{ \frac{(-1)^{n+1}}{n+2} \left(\frac{x+1}{3}\right)^{n+1}}
                       { \frac{(-1)^n}{n+1} \left(\frac{x+1}{3}\right)^n} \bigg| \\
&= \lim_{n\to\infty}\left|\frac{(-1)^{n+1}}{(-1)^n} \right|\cdot\left|\frac{n+1}{n+2} \right|\cdot\left|\frac{(x+1)^{n+1}}{(x+1)^n} \right|\cdot\left|\frac{3^n}{3^{n+1}} \right| \\
&= \lim_{n\to\infty} \left(\frac{n+1}{n+2}\right)\cdot \bigg| \frac{x+1}3 \bigg| = \frac{|x+1|}3
\end{align*}
Therefore, by the ratio test, the series converges when $\frac{|x+1|}3 < 1$
and diverges when $\frac{|x+1|}3 > 1$. In particular, it converges when
\begin{align*}
|x+1| < 3 \iff -3 < x+1 < 3 \iff -4 < x < 2
\end{align*}
and the radius of convergence is $R = 3$. (Alternatively, one can set $A_n = \frac{(-1)^n}{(n+1)3^n}$ and compute
$A = \lim_{n\to\infty} \big| \frac{A_{n+1}}{A_n}\big| = \frac{1}{3}$, so that
$R=\frac{1}{A}=3$.)

Next, we consider the endpoints $2$ and $-4$.
At $x=2$, i.e. $x+1=3$, the series is simply $\sum_{n=0}^\infty \frac{(-1)^{n}}{n+1}$,
which is an alternating series: the signs alternate, and the unsigned terms decrease to zero. Therefore the series converges at $x=2$ by the alternating series test.

At $x=-4$ the series is
$$\sum_{n=0}^\infty \frac{(-1)^{n}}{n+1} \left(\frac{-4+1}{3}\right)^n
 = \sum_{n=0}^\infty \frac{(-1)^{n}}{n+1} (-1)^n
 = \sum_{n=0}^\infty \frac1{n+1},$$
since $(-1)^n \cdot (-1)^n = (-1)^{2n} = \left((-1)^2\right)^n = 1$.
This series diverges, either by comparison or limit comparison with the harmonic series (the $p$-series with $p=1$). (For that matter, it is exactly equal to the standard harmonic
series $\sum_{n=1}^\infty \frac{1}{n}$, re-indexed to start at $n=0$.)

In summary, the interval of convergence is $-4<x\le2$, or simply $(-4,2]$.


\end{solution}



\begin{Mquestion}[2015A]
 Find the {\em interval of convergence} for the power series
\begin{align*}
\sum_{n=1}^\infty \frac{(x-2)^n}{n^{4/5}(5^n-4)}.
\end{align*}
\end{Mquestion}

%\begin{hint}
%\end{hint}

\begin{answer}
 $-3\le x< 7$ or $[-3,7)$
\end{answer}

\begin{solution}
We first apply the ratio test with $a_n = \frac{(x-2)^n}{n^{4/5}(5^n-4)}$.

\begin{align*}
\lim_{n\rightarrow\infty}\Big|\frac{a_{n+1}}{a_n}\Big|
&=\lim_{n\rightarrow\infty}\bigg|\frac{(x-2)^{n+1}}{(n+1)^{4/5}(5^{n+1}-4)}\cdot
                               \frac{n^{4/5}(5^n-4)}{(x-2)^n}\bigg|\\
&=\lim_{n\rightarrow\infty}\frac{n^{4/5}(5^n-4)}{(n+1)^{4/5}(5^{n+1}-4)}
                               |x-2|\\
&=\lim_{n\rightarrow\infty}\frac{(1-4/5^n)}{(1+1/n)^{4/5}(5-4/5^n)}
                               |x-2|\\
&=\frac{|x-2|}{5}
\end{align*}
Therefore the series converges if $|x-2|<5$ and diverges if $|x-2|>5$.
When $x-2=+5$, i.e. $x=7$, the series reduces to
$\sum\limits_{n=1}^\infty \frac{5^n}{n^{4/5}(5^n-4)}
 =\sum\limits_{n=1}^\infty \frac{1}{n^{4/5}(1-4/5^n)}$
which diverges by the limit comparison test with $b_n=\frac{1}{n^{4/5}}$.
When $x-2=-5$, i.e. $x=-3$, the series reduces to
$\sum\limits_{n=1}^\infty \frac{(-5)^n}{n^{4/5}(5^n-4)}
 =\sum\limits_{n=1}^\infty \frac{(-1)^n}{n^{4/5}(1-4/5^n)}$
which converges by the alternating series test. So the interval
of convergence is $-3\le x< 7$ or $[-3,7)$.

\end{solution}
%%%%%%%%%%%%%%%%%%%

\begin{question}[2016Q6]
Find all values $x$ for which the series
$\displaystyle\sum_{n=1}^\infty\frac{(x+2)^n}{n^2}$ converges.
\end{question}

%\begin{hint}
%\end{hint}

\begin{answer}
The given series converges if and only if  $-3\le x\le -1$.
Equivalently, the series has interval of convergence $[-3,-1]$.
\end{answer}

\begin{solution}
We apply the ratio test with $a_n=\frac{(x+2)^n}{n^2}$.
Since
\begin{equation*}
\lim_{n\rightarrow\infty}\Big|\frac{a_{n+1}}{a_n}\Big|
=\lim_{n\rightarrow\infty}\Bigg|\frac{\frac{(x+2)^{n+1}}{(n+1)^2}}
                                       {\frac{(x+2)^n}{n^2}}\Bigg|
=\lim_{n\rightarrow\infty}\frac{n^2}{{(n+1)}^2}|x+2|
=\lim_{n\rightarrow\infty}\frac{1}{{(1+1/n)}^2}|x+2|
=|x+2|
\end{equation*}
we have convergence for
\begin{equation*}
|x+2|<1
\iff -1<x+2<1
\iff -3<x<-1
\end{equation*}
and divergence for $|x+2|>1$. For $|x+2|=1$, i.e. for $x+2=\pm 1$,
i.e. for $x=-3,-1$, the series reduces to
$\sum\limits_{n=1}^\infty\frac{(\pm 1)^n}{n^2}$, which converges absolutely,
because $\sum\limits_{n=1}^\infty\frac{1}{n^p}$ converges for $p=2>1$.
So the given series converges if and only if  $-3\le x\le -1$.
\end{solution}
%%%%%%%%%%%%%%%%%%%

\begin{question}[2016Q6]
Find the interval of convergence for the following series.
\begin{enumerate}[(a)]
\item
$\displaystyle \sum_{n=1}^\infty \frac{4^n}{n}(x-1)^n$.
\item
$\displaystyle \sum_{n=1}^\infty \frac{4^n}{n}(2x+1)^n$.
\end{enumerate}
\end{question}

%\begin{hint}
%\end{hint}

\begin{answer}
(a) $\frac{3}{4}\le x<\frac{5}{4}$ or $\big[\frac{3}{4},\frac{5}{4}\big)$\qquad
(b) $-\frac{5}{8}\le x<-\frac{3}{8}$ or $\big[-\frac{5}{8},-\frac{3}{8}\big)$
\end{answer}

\begin{solution} (a)
We apply the ratio test with $a_n= \frac{4^n}{n}(x-1)^n$.
Since
\begin{align*}
\lim_{n\to\infty} \bigg| \frac{a_{n+1}}{a_n} \bigg|
&= \lim_{n\to\infty} \bigg| \frac{4^{n+1}(x-1)^{n+1}/(n+1)}{4^n(x-1)^n/n} \bigg| \\
&= \lim_{n\to\infty} 4|x-1| \frac{n}{n+1} \\
&= 4|x-1| \lim_{n\to\infty} \frac{n}{n+1} = 4|x-1|\cdot1.
\end{align*}
the series converges if
\begin{equation*}
4|x-1|<1
\iff -1<4(x-1)<1
\iff -\frac{1}{4}<x-1<\frac{1}{4}
\iff \frac{3}{4}<x<\frac{5}{4}
\end{equation*}
and diverges if $4|x-1|>1$.
Checking the right endpoint $x=\frac{5}{4}$, we see that
\begin{align*}
\sum_{n=1}^\infty \frac{4^n}{n}\bigg( \frac{5}{4}-1 \bigg)^n
= \sum_{n=1}^\infty \frac{1}{n}
\end{align*}
is the divergent harmonic series. At the left endpoint $x=\frac{3}{4}$,
\begin{align*}
\sum_{n=1}^\infty \frac{4^n}{n}\bigg( \frac34-1 \bigg)^n = \sum_{n=1}^\infty \frac{(-1)^n}{n}
\end{align*}
converges by the alternating series test. Therefore the interval of convergence of the
original series is $\frac{3}{4}\le x<\frac{5}{4}$, or $\big[\frac{3}{4},\frac{5}{4}\big)$.

\noindent (b)
We apply the ratio test with $a_n= \frac{4^n}{n}(2x+1)^n$.
Repeating the computation of part (a), just with $x-1$ replaced by $2x+1$, 
\begin{align*}
\lim_{n\to\infty} \bigg| \frac{a_{n+1}}{a_n} \bigg|
&= \lim_{n\to\infty} \bigg| \frac{4^{n+1}(2x+1)^{n+1}/(n+1)}{4^n(2x+1)^n/n} \bigg| \\
&= 4|2x+1| \lim_{n\to\infty} \frac{n}{n+1} = 4|2x+1|\cdot1.
\end{align*}
So the series converges if
\begin{align*}
4|2x+1|<1
&\iff -1<4(2x+1)<1 \\
&\iff -\frac{1}{4}<2x+1<\frac{1}{4} \\
&\iff -\frac{5}{4}<2x<-\frac{3}{4} \\
&\iff -\frac{5}{8}<x<-\frac{3}{8}
\end{align*}
and diverges if $4|2x+1|>1$.
At the right endpoint $x=-\frac{3}{8}$, the series becomes
\begin{align*}
\sum_{n=1}^\infty \frac{4^n}{n}\bigg( -\frac{3}{4}+1 \bigg)^n
= \sum_{n=1}^\infty \frac{1}{n}
\end{align*}
which is the divergent harmonic series. At the left endpoint $x=-\frac{5}{8}$,
\begin{align*}
\sum_{n=1}^\infty \frac{4^n}{n}\bigg( -\frac54+1 \bigg)^n = \sum_{n=1}^\infty \frac{(-1)^n}{n}
\end{align*}
converges by the alternating series test. Therefore the interval of convergence of the
original series is $-\frac{5}{8}\le x<-\frac{3}{8}$, or $\big[-\frac{5}{8},-\frac{3}{8}\big)$.

\end{solution}
%%%%%%%%%%%%%%%%%%%


\begin{Mquestion}[2012A]
Find, with explanation, the radius of convergence and the interval of
convergence of the power series
\begin{equation*}
\sum_{n=0}^\infty (-1)^n\frac{(x-1)^n}{2^n(n+2)}
\end{equation*}
\end{Mquestion}

%\begin{hint}
%\end{hint}

\begin{answer}
The radius of convergence is $2$.
The interval of convergence  is $-1< x\le3$,
or $\big(-1,3\big]$.
\end{answer}

\begin{solution}
We apply the ratio test with $a_n= (-1)^n\frac{(x-1)^n}{2^n(n+2)}$.
Since
\begin{align*}
\lim_{n\to\infty} \bigg| \frac{a_{n+1}}{a_n} \bigg|
&= \lim_{n\to\infty} \bigg| \frac{(x-1)^{n+1}}{2^{n+1}(n+3)}\
                            \frac{2^n(n+2)}{(x-1)^n} \bigg| \\
&= \lim_{n\to\infty} \frac{|x-1|}{2} \frac{n+2}{n+3} \\
&= \frac{|x-1|}{2} \lim_{n\to\infty} \frac{1+2/n}{1+3/n} = \frac{|x-1|}{2}
\end{align*}
the series converges if
\begin{align*}
\frac{|x-1|}{2}<1
&\iff |x-1|<2
\iff -2<(x-1)<2
\iff -1<x<3
\end{align*}
and diverges if $|x-1|>2$. So the series has radius of convergence $2$.
Checking the left endpoint $x=-1$, so that $\frac{x-1}{2}=-1$,
we see that
\begin{align*}
\sum_{n=0}^\infty (-1)^n\frac{(-1-1)^n}{2^n(n+2)}
= \sum_{n=0}^\infty \frac{1}{n+2}
\end{align*}
is the divergent harmonic series. At the right endpoint $x=3$,
so that $\frac{x-1}{2}=+1$ and
\begin{align*}
\sum_{n=0}^\infty (-1)^n\frac{(3-1)^n}{2^n(n+2)}
= \sum_{n=0}^\infty \frac{(-1)^n}{n+2}
\end{align*}
converges by the alternating series test. Therefore the interval of convergence of the
original series is $-1< x\le 3$, or $\big(-1,3\big]$.

\end{solution}
%%%%%%%%%%%%%%%%%%%


\begin{question}[2014D]
Find the \emph{interval of convergence} for the series
$\displaystyle \sum_{n=1}^\infty (-1)^n n^2(x-a)^{2n}$
where $a$ is a constant.
\end{question}

%\begin{hint}
%\end{hint}

\begin{answer}
The interval of convergence  is $\ \ a-1< x<a+1$,
or $\big(a-1,a+1\big)$.
\end{answer}

\begin{solution}
We apply the ratio test with $a_n= (-1)^n n^2(x-a)^{2n}$.
Since
\begin{align*}
\lim_{n\to\infty} \bigg| \frac{a_{n+1}}{a_n} \bigg|
&= \lim_{n\to\infty} \bigg| \frac{(-1)^{n+1} (n+1)^2(x-a)^{2(n+1)}}
                                 {(-1)^n n^2(x-a)^{2n}} \bigg| \\
&= \lim_{n\to\infty} |x-a|^2 \frac{(n+1)^2}{n^2} \\
&= |x-a|^2 \lim_{n\to\infty} \big(1+\nicefrac{1}{n}\big)^2 = |x-a|^2\cdot1.
\end{align*}
the series converges if
\begin{equation*}
|x-a|^2<1
\iff |x-a|<1
\iff-1<x-a<1
\iff a-1<x<a+1
\end{equation*}
and diverges if $|x-a|>1$.
Checking both endpoints $x-a=\pm 1$, we see that
\begin{align*}
 \sum_{n=1}^\infty (-1)^n n^2(x-a)^{2n}\bigg|_{x-a=\pm 1}
=\ \sum_{n=1}^\infty (-1)^n n^2
\end{align*}
fails the divergence test --- the $n^{\rm th}$ term does not converge
to zero as $n\rightarrow\infty$. Therefore the interval of convergence
of the original series is  $a-1< x<a+1$,  or $\big(a-1,a+1\big)$.

\end{solution}
%%%%%%%%%%%%%%%%%%%




\begin{Mquestion}[2015A]
 Find the interval of convergence of the following series:
\begin{enumerate}[(a)]
\item
${\displaystyle \sum_{k=1}^\infty \frac{(x+1)^k}{k^2 9^k}}$.
\item
${\displaystyle \sum_{k=1}^\infty a_k(x-1)^k}$, \ where
$a_k>0$ for $k=1,2,\cdots$ and
$\ {\displaystyle \sum_{k=1}^\infty \Big(\frac{a_k}{a_{k+1}}
                         -\frac{a_{k+1}}{a_{k+2}}\Big)
   =\frac{a_1}{a_2} }$.
\end{enumerate}
\end{Mquestion}

\begin{hint}
Start part (b) by computing the partial sums of
$\ {\displaystyle \sum_{k=1}^\infty \Big(\frac{a_k}{a_{k+1}}
                         -\frac{a_{k+1}}{a_{k+2}}\Big)}$
\end{hint}

\begin{answer}
(a)  $|x+1|\le 9$ or $-10\le x\le 8$ or $[-10,8]$
\qquad (b) This series converges only for $x=1$.
\end{answer}

\begin{solution} (a)
We apply the ratio test for the series whose $k^{\rm th}$
term is $A_k=\frac{(x+1)^k}{k^2 9^k}$. Then
\begin{align*}
\lim_{k\rightarrow\infty}\bigg|\frac{A_{k+1}}{A_k}\bigg|
&=\lim_{k\rightarrow\infty} \bigg|\frac{(x+1)^{k+1}}{(k+1)^2 9^{k+1}}
                                  \ \frac{k^2 9^k}{(x+1)^k}\bigg| \\
&=\lim_{k\rightarrow\infty} |x+1|\ \frac{1}{9}\ \frac{k^2}{(k+1)^2} \\
&=\lim_{k\rightarrow\infty} |x+1|\ \frac{1}{9}\ \frac{1}{(1+1/k)^2} \\
&=\frac{|x+1|}{9}
\end{align*}
So the series must converge when $|x+1|<9$ and must diverge when $|x+1|>9$.
When $x+1=\pm 9$, the series reduces to
\begin{align*}
\sum_{k=1}^\infty \frac{(\pm 9)^k}{k^2 9^k}
=\sum_{k=1}^\infty \frac{(\pm 1)^k}{k^2}
\end{align*}
which converges (since, by the $p$--test, $\sum_{k=1}^\infty\frac{1}{k^p}$ converges for
any $p>1$). So the interval of convergence is $|x+1|\le 9$
or $-10\le x\le 8$  or $[-10,8]$.



\noindent (b)
The partial sum
\begin{align*}
\sum_{k=1}^N\Big(\frac{a_k}{a_{k+1}}
                         -\frac{a_{k+1}}{a_{k+2}}\Big)
=\Big(\frac{a_1}{a_2}-\frac{a_2}{a_3}\Big)
 +\Big(\frac{a_2}{a_3}-\frac{a_3}{a_4}\Big)
+\cdots+ \Big(\frac{a_N}{a_{N+1}}-\frac{a_{N+1}}{a_{N+2}}\Big)
=\frac{a_1}{a_2}-\frac{a_{N+1}}{a_{N+2}}
\end{align*}
We are told that ${\displaystyle \sum_{k=1}^\infty \Big(\frac{a_k}{a_{k+1}}
                         -\frac{a_{k+1}}{a_{k+2}}\Big)
   =\frac{a_1}{a_2} }$. This means that the above partial sum converges to
$\frac{a_1}{a_2}$ as $N\rightarrow\infty$, or equivalently, that
\begin{align*}
\lim_{N\rightarrow\infty}\frac{a_{N+1}}{a_{N+2}}=0
\end{align*}
and hence that
\begin{align*}
\lim_{k\rightarrow\infty}\frac{|a_{k+1}(x-1)^{k+1}|}{|a_k(x-1)^k|}
=|x-1|\lim_{k\rightarrow\infty}\frac{|a_{k+1}|}{|a_k|}
\end{align*}
is infinite for any $x\ne 1$. So, by the ratio test, this series converges only for $x=1$.
\end{solution}
%%%%%%%%%%%%%%%%%%%


\begin{Mquestion}[2014A]
Find a power series representation for $\dfrac{x^3}{1-x}$.
\end{Mquestion}

\begin{hint}
You should know a power series representation for $\dfrac{1}{1-x}$.
Use it.
\end{hint}

\begin{answer}
$\displaystyle\sum\limits_{n=0}^\infty x^{n+3}
=\sum\limits_{n=3}^\infty x^{n}$
\end{answer}

\begin{solution}
Using the geometric series $\sum\limits_{n=0}^\infty x^n = \frac{1}{1-x}$,
\begin{align*}
\frac{x^3}{1-x}
=x^3 \sum_{n=0}^\infty x^n
=\sum_{n=0}^\infty x^{n+3}
=\sum_{n=3}^\infty x^{n}
\end{align*}
\end{solution}
%%%%%%%%%%%%%%%%%%%



\begin{Mquestion}
	Suppose $f'(x)=\displaystyle\sum_{n=0}^\infty \frac{(x-1)^{n}}{n+2}$, and
	$\displaystyle \int_5^x f(t)\dee{t}=3x+\displaystyle\sum_{n=1}^\infty \frac{(x-1)^{n+1}}{n(n+1)^2}$.

	Give a power series representation of $f(x)$.
\end{Mquestion}
\begin{hint}
	You can safely ignore one of the given equations, but not the other.
\end{hint}
\begin{answer}
	$f(x)=3+\displaystyle\sum_{n=1}^\infty \frac{(x-1)^{n}}{n(n+1)}$
\end{answer}
\begin{solution}
	We can find $f(x)$ by differentiating its integral, or antidifferentiating its derivative. In the latter case, we'll have to solve for the arbitrary constant of integration; in the former case, we do not. (Remember that many different functions have the same derivative, but a single function has only one derivative.) To avoid the necessity of finding the arbitrary constant, we can ignore the given equation for $f'(x)$, which makes the problem much simpler. This is the method used in Solution 1.
	\begin{description}
		\item[Solution 1]
		Using the Fundamental Theorem of Calculus Part 1:
		\begin{align*}
		\diff{}{x} \left\{\int_5^x f(t)\dee{t} \right\}&=f(x)\\
		\text{So,}\qquad
		f(x)&=\diff{}{x} \left\{3x+\displaystyle\sum_{n=0}^\infty \frac{(x-1)^{n+1}}{n(n+1)^2}  \right\}\\
		&=3+\displaystyle\sum_{n=1}^\infty \frac{(n+1)(x-1)^{n}}{n(n+1)^2}\\
		&=3+\displaystyle\sum_{n=1}^\infty \frac{(x-1)^{n}}{n(n+1)}
		\end{align*}

		\item[Solution 2]
		Suppose we had used $f'(x)$ instead. We would antidifferentiate to find:
		\begin{align*}
		f(x)&=\int \left( \sum_{n=0}^\infty \frac{(x-1)^{n}}{n+2}  \right)\dee{x}\\
		&=\left(\sum_{n=0}^\infty \frac{(x-1)^{n+1}}{(n+1)(n+2)}\right)+C\\
		&=\left(\sum_{n=1}^\infty \frac{(x-1)^{n}}{n(n+1)}\right)+C
		\end{align*}
		Notice $f(1)=0+C$. So, to find $C$, we must find $f(1)$. We can't get that information from $f'(x)$, so our only option is to consider the given formula for $\int_5^x f(t)\dee{t}$. Using the Fundamental Theorem of Calculus Part 1:
		\begin{align*}
		f(1)&=\left.\diff{}{x}\left\{\int_5^x f(t)\dee{t} \right\}\right|_{x=1}\\
		&=\left.\diff{}{x}\left\{ 3x+\displaystyle\sum_{n=1}^\infty \frac{(x-1)^{n+1}}{n(n+1)^2}  \right\}\right|_{x=1}\\
		&=\left[ 3+\displaystyle\sum_{n=1}^\infty \frac{(n+1)(x-1)^{n}}{n(n+1)^2}  \right]_{x=1}\\
		&=\left[ 3+\displaystyle\sum_{n=1}^\infty \frac{(x-1)^{n}}{n(n+1)}  \right]_{x=1}\\
		&= 3+\displaystyle\sum_{n=1}^\infty \frac{0^{n}}{n(n+1)}\\
		&=3
		\end{align*}
		So, $f(x)=3+\displaystyle\sum_{n=1}^\infty \frac{(x-1)^{n}}{n(n+1)}$.

		Note that in Solution 2, we did the same calculation as Solution 1, and more.
	\end{description}
\end{solution}


%%%%%%%%%%%%%%%%%%
\subsection*{\Application}
%%%%%%%%%%%%%%%%%%

\begin{Mquestion}[M121 2002A]
Determine the values of $x$ for which the series
\begin{equation*}
\sum_{n=2}^\infty\frac{x^n}{3^{2n}\log n}
\end{equation*}
converges absolutely, converges conditionally, or diverges.
\end{Mquestion}

\begin{hint}
$n\ge\log n$ for all $n\ge 1$.
\end{hint}

\begin{answer}
The series converges absolutely for $|x|<9$, converges conditionally
for $x=-9$ and diverges otherwise.
\end{answer}

\begin{solution}
We apply the ratio test for the series whose $n^{\rm th}$
term is either $a_n=\frac{x^n}{3^{2n}\log n}$
or $a_n=\big|\frac{x^n}{3^{2n}\log n}\big|$. For both series
\begin{align*}
\lim_{n\to\infty} \bigg| \frac{ a_{n+1} }{ a_n } \bigg|
&= \lim_{n\to\infty} \bigg| \frac{x^{n+1}}{3^{2(n+1)}\log(n+1)} \
                               \frac{3^{2n}\log n}{x^n} \bigg| \\
&= \lim_{n\to\infty} \bigg| \frac{x\log n}{3^2\log(n+1)} \bigg|
=\lim_{n\to\infty} \bigg| \frac{x\log n}{3^2[\log(n)+\log(1+1/n)]} \bigg| \\
&=\lim_{n\to\infty} \bigg| \frac{x}{3^2[1+\log(1+1/n)/\log(n)]} \bigg| \\
&=\frac{|x|}{9}
\end{align*}
Therefore, by the ratio test, our series converges absolutely when
$|x|<9$ and diverges when $|x|>9$.

For $x=-9$, $\displaystyle\sum_{n=2}^\infty\frac{x^n}{3^{2n}\log n}
=\sum_{n=2}^\infty\frac{(-1)^n}{\log n}$ which converges by
the alternating series test.

For $x=+9$, $\displaystyle\sum_{n=2}^\infty\frac{x^n}{3^{2n}\log n}
=\sum_{n=2}^\infty\frac{1}{\log n}$ which is the same series
as $\displaystyle\sum_{n=2}^\infty\Big|\frac{(-1)^n}{\log n}\Big|$.
We shall shortly
show that $n\ge\log n$, and hence $\frac{1}{\log n}\ge \frac{1}{n}$
for all $n\ge 1$.
This implies that the series $\displaystyle\sum_{n=2}^\infty\frac{1}{\log n}$
diverges by comparison with the divergent series
$\displaystyle\sum_{n=2}^\infty\frac{1}{n^p}\bigg|_{p=1}$.
This yelds both divergence for $x=9$ and
also the failure of absolute convergence for $x=-9$.

Finally, we show that $n-\log n > 0$, for all $n\ge 1$.
Set $f(x)=x-\log x$. Then $f(1)=1>0$ and
\begin{equation*}
f'(x) = 1 -\frac{1}{x} \ge 0\qquad\text{for all }x\ge 1
\end{equation*}
So $f(x)$ is (strictly) positive when $x=1$ and is increasing for all
$x\ge 1$. So $f(x)$ is (strictly) positive for all $x\ge 1$.


\end{solution}
%%%%%%%%%%%%%%%%%%%




\begin{Mquestion}[2013A]
 (a)  Find the power--series representation for
$\displaystyle\int\frac{1}{1+x^3}\,\dee{x}$ centred at $0$ (i.e. in powers of
$x$).

\noindent (b) The power series above is used to approximate
$\displaystyle\int_0^{1/4}\frac{1}{1+x^3}\,\dee{x}$. How many terms are
required to guarantee that the resulting approximation is within
$10^{-5}$ of the exact value? Justify your answer.
\end{Mquestion}

\begin{hint}
See Example \eref{CLP101}{eg:SRpsrepD} in the
%\href{http://www.math.ubc.ca/%7Efeldman/m101/clp/clp_notes_101.pdf}{CLP-2 text}.
CLP-2 text.
For part (b), review  \S \eref{CLP101}{sec:AST} in the
%\href{http://www.math.ubc.ca/%7Efeldman/m101/clp/clp_notes_101.pdf}{CLP-2 text}.
CLP-2 text.
\end{hint}

\begin{answer}
(a)
$\displaystyle\sum\limits_{n=0}^\infty(-1)^n \frac{x^{3n+1}}{3n+1} +C$
\quad (b)
We need to keep two terms (the $n=0$ and $n=1$ terms).
\end{answer}

\begin{solution} (a)
Applying $\displaystyle\frac{1}{1+r}=\sum\limits_{n=0}^\infty (-1)^nr^n$ with $r=x^3$
gives
\begin{align*}
\int\frac{1}{1+x^3}\ \dee{x} =\sum_{n=0}^\infty(-1)^n \int x^{3n}\ \dee{x}
=\sum_{n=0}^\infty(-1)^n \frac{x^{3n+1}}{3n+1} +C
\end{align*}


\noindent (b) By part (a),
\begin{align*}
\int_0^{1/4}\frac{1}{1+x^3}\ \dee{x}
=\sum_{n=0}^\infty(-1)^n \frac{x^{3n+1}}{3n+1} \bigg|_0^{1/4}
=\sum_{n=0}^\infty(-1)^n \frac{1}{(3n+1)4^{3n+1}}
\end{align*}
This is an alternating series with successively smaller terms that converge
to zero as $n\rightarrow\infty$. So truncating
it introduces an error no larger than the magnitude of the first dropped
term. We want that first dropped term to obey
\begin{align*}
\frac{1}{(3n+1)4^{3n+1}}<10^{-5}=\frac{1}{10^5}
\end{align*}
So let's check the first few terms.
\begin{align*}
\frac{1}{(3n+1)4^{3n+1}}\bigg|_{n=0}&=\frac{1}{4}>\frac{1}{10^5}\\
\frac{1}{(3n+1)4^{3n+1}}\bigg|_{n=1}&=\frac{1}{4^5}>\frac{1}{10^5}\\
\frac{1}{(3n+1)4^{3n+1}}\bigg|_{n=2}&=\frac{1}{7\times 4^7}
                         =\frac{1}{7\times 2^{14}}
                         =\frac{1}{7\times16\times 1024}
                         =\frac{1}{112\times 1024}<\frac{1}{10^5}
\end{align*}
So we need to keep two terms (the $n=0$ and $n=1$ terms).

\end{solution}
%%%%%%%%%%%%%%%%%%%


\begin{question}[2014A]
 (a)  Show that $\displaystyle\sum_{n=0}^\infty nx^n =\frac{x}{(1-x)^2}$
for $-1<x<1$.

\noindent (b) Express $\displaystyle\sum_{n=0}^\infty n^2x^n$ as a
ratio of polynomials. For which $x$ does this series converge?
\end{question}

\begin{hint}
You know the geometric series expansion of $\frac{1}{1-x}$. What (calculus)
operation(s) can you apply to that geometric series to convert it into
the given series?
\end{hint}

\begin{answer}
(a)
See the solution.

\noindent  (b)
$\displaystyle\sum\limits_{n=0}^\infty n^2 x^n=\frac{x(1+x)}{(1-x)^3}$.
The series converges for $-1<x<1$.
\end{answer}

\begin{solution} (a)
Differentiating both sides of
\begin{equation*}
\sum_{n=0}^\infty x^n = \frac{1}{1-x}
\end{equation*}
gives
\begin{equation*}
\sum_{n=0}^\infty n x^{n-1}=\frac{1}{(1-x)^2}
\end{equation*}
Now multiplying both sides by $x$ gives
\begin{equation*}
\sum_{n=0}^\infty n x^n=\frac{x}{(1-x)^2}
\end{equation*}
as desired.


\noindent (b) Differentiating both sides of the conclusion of part (a) gives
\begin{equation*}
\sum_{n=0}^\infty n^2 x^{n-1}=\frac{(1-x)^2-2x(x-1)}{(1-x)^4}
=\frac{(1-x)(1-x+2x)}{(1-x)^4}
=\frac{1+x}{(1-x)^3}
\end{equation*}
Now multiplying both sides by $x$ gives
\begin{equation*}
\sum_{n=0}^\infty n^2 x^n=\frac{x(1+x)}{(1-x)^3}
\end{equation*}
We know that differentiation preserves the radius
of convergence of power series. So this series has radius of
convergence $1$ (the radius of convergence of the original
geometric series). At $x=\pm1 $ the series diverges by the divergence
test. So the series converges for $-1<x<1$.


\end{solution}
%%%%%%%%%%%%%%%%%%%

\begin{Mquestion}[2015A]
 Suppose that you have a sequence $\{b_n\}$ such that the series $\sum_{n=0}^{\infty}(1-b_n)$ converges. Using the tests we've learned in class, prove that the radius of convergence of the power series $\displaystyle\sum_{n=0}^{\infty}b_nx^n$ is equal to~$1$.
\end{Mquestion}

\begin{hint}
First show that the fact that the series $\sum_{n=0}^{\infty}(1-b_n)$ converges
 guarantees that $\lim_{n\rightarrow\infty}b_n=1$.
\end{hint}

\begin{answer}
See the solution.
\end{answer}

\begin{solution}
By the divergence test, the fact that $\sum\limits_{n=0}^{\infty}(1-b_n)$
converges guarantees that $\lim\limits_{n\rightarrow\infty}(1-b_n)=0$,
or equivalently, that $\lim\limits_{n\rightarrow\infty}b_n=1$.  So, by
equation  (\eref{CLP101}{eq:SRradConv})  in the
%\href{http://www.math.ubc.ca/%7Efeldman/m101/clp/clp_notes_101.pdf}{CLP-2 text}.
CLP-2 text, the radius of convergence is
\begin{equation}\label{eq:SRradConv}
R =\bigg[\lim_{n\rightarrow\infty}\Big|\frac{b_{n+1}}{b_n}\Big|\bigg]^{-1}
   =\bigg[\frac{1}{1}\bigg]^{-1}
  =1
\end{equation}


%So,
%by the limiting comparison test, the series $\sum\limits_{n=0}^{\infty}b_nx^n$
%converges if and only if the series $\sum\limits_{n=0}^{\infty}x^n$
%converges, which is the case if and only if $|x|< 1$.  So the
%radius of convergence is indeed $1$.
\end{solution}
%%%%%%%%%%%%%%%%%%%

\begin{question}[M121 2014A]
Assume $\big\{a_n \big\}$ is a sequence such that $na_n$ decreases to $C$ as
$n \rightarrow\infty$ for some real number $C > 0$


\noindent (a) Find the radius of convergence of $\displaystyle\sum\limits_{n=1}^\infty
 a_n x^n$ . Justify your answer carefully.

\noindent (b)  Find the interval of convergence of the above power series, that is,
find all $x$ for which the power series in (a) converges. Justify your answer carefully.
\end{question}

\begin{hint}
What does $a_n$ look like for large $n$?
\end{hint}

\begin{answer}
(a) $1$.\qquad
(b) The series converges for $-1\le x<1$, i.e. for the interval $[-1,1)$
\end{answer}

\begin{solution}
(a) We know that the radius of convergence $R$ obeys
\begin{align*}
\frac{1}{R} = \lim_{n\rightarrow\infty}\frac{a_{n+1}}{a_n}
                     = \lim_{n\rightarrow\infty}\frac{n}{n+1}\ \frac{(n+1)a_{n+1}}{na_n}
                     = 1 \frac{C}{C}
                     = 1
\end{align*}
because we are told that $\lim\limits_{n\rightarrow\infty} na_n=C$. So $R=1$.

\noindent (b) Just knowing that the radius of convergence is $1$, we know that the
series converges for $|x|<1$ and diverges for $|x|>1$. That leaves $x\pm 1$.

\noindent When $x=+1$, the series reduces to $\sum\limits_{n=1}^\infty a_n $. We are told that
$na_n$ \emph{decreases} to $C>0$. So $a_n\ge \frac{C}{n}$. By the comparison test
with the harmonic series $\sum\limits_{n=1}^\infty\frac{1}{n}$, which diverges by the $p$--test
with $p=1$, our series diverges when $x=1$.

\noindent When $x=-1$, the series reduces to  $\sum\limits_{n=1}^\infty (-1)^n a_n $.
We are told that $na_n$ \emph{decreases} to $C>0$. So $a_n>0$ and $a_n$ converges to
$0$ as $n\rightarrow\infty$. Consequently $\sum\limits_{n=1}^\infty (-1)^n a_n $
converges by the alternating series test.

\noindent In conclusion  $\sum\limits_{n=1}^\infty a_n x^n$ converges when $-1\le x< 1$.
\end{solution}
%%%%%%%%%%%%%%%%%%%



\begin{Mquestion}
An infinitely long, straight rod of negligible mass has the following weights:
\begin{itemize}
\item At every whole number $n$, a mass of weight $\dfrac{1}{2^n}$ at position $n$, and
\item a mass of weight $\dfrac{1}{3^n}$ at position $-n$.
\end{itemize}
At what position is the centre of mass of the rod?
\begin{center}
\begin{tikzpicture}
\draw[help lines, <->] (-6.5,0)--(7,0);
\YEnxcoord{0}{0}
\foreach \l in {1,2,3}{
	\MULTIPLY{\l}{2}{\x}
	\YEnxcoord{\x}{\l}
	\YEnxcoord{-\x}{-\l}
	\POWER{2}{\l}{\q}
	\DIVIDE{3}{\q}{\m}
	\POWER{3}{\l}{\Q}
	\DIVIDE{3}{\Q}{\M}
	\draw[fill, fill opacity=0.2] (\x,0) arc(90:450:\m cm);
	\draw[fill, fill opacity=0.2] (-\x,0) arc(90:450:\M cm);
\draw (\x,\x/2-4.5) node[below]{$\frac{1}{2^{\l}}$};
\draw (-\x,\x/2-3.5) node[below]{$\frac{1}{3^{\l}}$};
}
\end{tikzpicture}
\end{center}

\end{Mquestion}
\begin{hint}
Equation~\eref{CLP101}{eq:weightedrod} in the CLP-2 text tells us the centre of mass of a rod with weights $\{m_n\}$ at positions $\{x_n\}$ is $\displaystyle\bar x =\frac{\sum m_nx_n}{\sum m_n}$ .
\end{hint}
\begin{answer}
	$\dfrac{5}{6}$
\end{answer}
\begin{solution}
Equation~\eref{CLP101}{eq:weightedrod} in the CLP-2 text tells us the centre of mass of a rod with weights $\{m_n\}$ at positions $\{x_n\}$ is $\displaystyle\bar x =\frac{\sum m_nx_n}{\sum m_n}$ .

We find the combined mass of our weights using Equation~\eref{CLP101}{eq:geomsum}
 in the CLP-2 text with $r=\frac{1}{2}$ and $r=\frac{1}{3}$, respectively.
\begin{align*}
\sum_{n=1}^{\infty}\frac{1}{2^n}+\sum_{n=1}^{\infty}\frac{1}{3^n}&=
\sum_{n=0}^{\infty}\frac{1}{2}\cdot\frac{1}{2^n}+\sum_{n=0}^{\infty}\frac{1}{3}\cdot\frac{1}{3^n}\\
&=\frac{1}{2}\cdot\frac{1}{1-\frac12}+\frac{1}{3}\cdot\frac{1}{1-\frac13}\\
&=1+\frac12=\frac32
\end{align*}

Now, we want to calculate the sum of the products of the masses and their positions.
\[\sum_{n=1}^{\infty}\frac{1}{2^n}\cdot n
       +\sum_{n=1}^{\infty}\frac{1}{3^n}\cdot (-n)\]
We don't have such a nice formula for this, but we can make one by differentiating.

The following formula is true for any $x$ with $|x|<1$:
\begin{align*}
\sum_{n=0}^\infty x^n&=\frac{1}{1-x}
\intertext{Differentiating both sides with respect to $x$:}
\sum_{n=0}^\infty nx^{n-1}&=\frac{1}{(1-x)^2}\\
\sum_{n=1}^\infty nx^{n-1}&=\frac{1}{(1-x)^2}
\intertext{Multiplying both sides by $x$:}
\sum_{n=1}^\infty nx^{n}&=\frac{x}{(1-x)^2}
\end{align*}

This allows us to evaluate our series.
\begin{align*}
\sum_{n=1}^{\infty}\frac{n}{2^n}-\sum_{n=1}^{\infty}\frac{n}{3^n}
&=\frac{\frac12}{\left(1-\frac12\right)^2}-\frac{\frac13}{\left(1-\frac13\right)^2}\\
&=2-\frac{3}{4} = \frac{5}{4}
\intertext{Therefore,}
\bar x &=\frac{5/4}{3/2}=\frac{5}{6} = 0.8\overline{33}
\end{align*}
Remark: we can check that this makes some sense. Since the weights to the right of $x=0$ are heavier than those to the left, but spaced the same, we would expect our rod to balance to the right of $x=0$.
\end{solution}
%%%%%%%%%%%%%%%%%%%


\begin{question}
	Let $f(x)=\displaystyle\sum_{n=0}^{\infty}A_n(x-c)^n$, for some constant $c$ and a sequence of constants $\{A_n\}$. Further, let $f(x)$ have a positive radius of convergence.

	If $A_1=0$, show that $y=f(x)$ has a critical point at $x=c$. What is the relationship between the behaviour of the graph at that point and the value of $A_2$?
\end{question}
\begin{hint}
	Use the second derivative test.
\end{hint}
\begin{answer}
	The point $x=c$ corresponds to a local maximum if $A_2<0$ and a local minimum if $A_2>0$.
\end{answer}
\begin{solution}
	First, we differentiate.
	\begin{align*}
	f(x)&=\displaystyle\sum_{n=0}^{\infty}A_n(x-c)^n\\
	f'(x)&=\sum_{n=0}^{\infty}nA_n(x-c)^{n-1}\\
	&=\sum_{n=1}^{\infty}nA_n(x-c)^{n-1}\\
	f'(c)&=\sum_{n=1}^\infty nA_n\cdot 0^{n-1}\\
	&=A_1\cdot 1 + 2A_2\cdot0+3A_3\cdot0+\cdots\\
	&=A_1
	\end{align*}
	So, if $A_1=0$, then $f'(c)=0$. That is, $f(x)$ has a critical point at $x=c$.

	To determine the behaviour of this critical point, we use the second derivative test.
	\begin{align*}
	f'(x)&=\sum_{n=1}^{\infty}nA_n(x-c)^{n-1}\\
	f''(x)&=\sum_{n=1}^{\infty}n(n-1)A_n(x-c)^{n-2}\\
	&=\sum_{n=2}^{\infty}n(n-1)A_n(x-c)^{n-2}\\
	f''(c)&=\sum_{n=2}^{\infty}n(n-1)A_n\cdot 0^{n-2}\\
	&=2(1)A_2\cdot 0^0+3(2)A_3\cdot 0^1+4(3)A_4\cdot 0^2+\cdots\\
	&=2A_2
	\end{align*}
	Following the second derivative test, $x=c$	is the location of a local maximum if $A_2<0$, and it is the location of a local minimum if $A_2>0$. (If $A_2=0$, the critical point may or may not  be a local extremum.)
\end{solution}


\begin{Mquestion}
	Evaluate $\displaystyle \sum_{n=3}^\infty \frac{n}{5^{n-1}}$.
\end{Mquestion}
\begin{hint}
	What function has $\displaystyle\sum_{n=1}^\infty nx^{n-1}$ as its power series representation?
\end{hint}
\begin{answer}
	$\dfrac{13}{80}$
\end{answer}
\begin{solution}
	We recognize
	$\displaystyle \sum_{n=3}^\infty \frac{n}{5^{n-1}}$ as
	$f(x)=\displaystyle \sum_{n=3}^\infty n\cdot x^{n-1}$, evaluated at $x=\dfrac15$. We should figure out what $f(x)$ is in equation form (as opposed to power series form). Notice that this looks similar to the derivative of the geometric series $\displaystyle\sum x^n$.

	\begin{align*}
	\frac{1}{1-x}&=\sum_{n=0}^\infty x^n \qquad \text{when $|x|<1$}\\
	\diff{}{x}\left\{\frac{1}{1-x}\right\}&=\diff{}{x}\left\{\sum_{n=0}^\infty x^n\right\}\\
	\frac{1}{(1-x)^2}&=\sum_{n=1}^\infty nx^{n-1}\\
	&= 1x^0 + 2x^1+\sum_{n=3}^\infty nx^{n-1}\\
	&= 1 + 2x+\sum_{n=3}^\infty nx^{n-1}\\
	\text{So,}\qquad
	\frac{1}{(1-x)^2}-1-2x&=\sum_{n=3}^\infty nx^{n-1}\\
	\text{Setting $x=\dfrac15$:}\qquad
	\frac{1}{(1-1/5)^2}-1-\frac25&=\sum_{n=3}^\infty n\left(\frac15\right)^{n-1}\\
	\left(\frac{5}{4} \right)^2-1-\frac25&=\sum_{n=3}^\infty \frac{n}{5^{n-1}}\\
	\end{align*}
	So,  our series evaluates to $\displaystyle\frac{25}{16}-1-\frac{2}{5}=\frac{13}{80}$.
\end{solution}
%%%%%%%%%%%%%%%%%%%%%%%%%
\begin{Mquestion}
Find a polynomial that approximates $f(x)=\log(1+ x)$  to within an error of $10^{-5}$ for all values of $x$ in $\left(0,\frac{1}{10}\right)$.

Then, use your polynomial to approximate $\log(1.05)$ as a rational number.
\end{Mquestion}
\begin{hint}
The power series representation in Example~\eref{CLP101}{eg:SRpsrepC} is an alternating series when $x$ is positive.
\end{hint}
\begin{answer}
$x-\dfrac{x^2}{2}+\dfrac{x^3}{3}-\dfrac{x^4}{4}$
\end{answer}
\begin{solution}
As we saw in in Example~\eref{CLP101}{eg:SRpsrepC}  of the CLP-2 text,
\[\log(1+ x) = \sum_{n=0}^\infty (-1)^n\frac{x^{n+1}}{n+1}\]
which is an alternating series when $x$ is positive. If we use its partial sum $S_N$ to approximate $\log(1+ x)$, the absolute error involved is no more than
 \[\frac{x^{(N+1)+1}}{(N+1)+1} =\frac{x^{N+2}}{N+2}\]
 We want this error to be at most $10^{-5}$ whenever $0< x <\frac1{10}$. For this range of $x$ values,
$ \dfrac{x^{N+2}}{N+2}<\dfrac{1}{(N+2)10^{N+2}}$, so we want $N$ that satisfies the  inequality:
\begin{align*}
\frac{1}{(N+2)10^{N+2}}&\le \frac{1}{10^5}\\
\Rightarrow \qquad (N+2)10^{N+2}&\ge 10^5
\end{align*}
We see $N=3$ suffices.

So, the partial sum
\[ \sum_{n=0}^3 (-1)^n\frac{x^{n+1}}{n+1}=x-\frac{x^2}{2}+\frac{x^3}{3}-\frac{x^4}{4}\]
approximates $\log(1+x)$ to within an error of
$\dfrac{x^5}{5}$.

When $x$ is between $0$ and $\frac1{10}$,
that error is at most $\dfrac{1}{5\cdot 10^5}<10^{-5}$, as desired.

Now we can approximate $\log(1.05)$.
\begin{align*}
\log(1.05)&=\log\left(1+\frac{1}{20}\right)\\
&\approx \left(\tfrac{1}{20}\right)-\dfrac{\left(\tfrac{1}{20}\right)^2}{2}+\dfrac{\left(\tfrac{1}{20}\right)^3}{3}-\dfrac{\left(\tfrac{1}{20}\right)^4}{4}\\
&=\frac{12\times 20^3-6\times 20^2+4\times 20-3}{12\times 20^4}=\frac{93677}{1920000}
\end{align*}

We note that a computer approximates
$\frac{93677}{1920000}\approx0.04879010$ and
$\log(1.05) \approx 0.04879016$. So, our actual error is around $6\times 10^{-8}$.
\end{solution}

%%%%%%%%%%%%%%%%%%%%%%%%%
\begin{question}
Find a polynomial that approximates $f(x)=\arctan x$  to within an error of $10^{-5}$ for all values of $x$ in $\left(-\frac{1}{4},\frac{1}{4}\right)$.
\end{question}
\begin{hint}
The power series representation in Example~\eref{CLP101}{eg:SRpsrepD} is an alternating series when $x$ is nonzero.
\end{hint}
\begin{answer}
$x-\dfrac{x^3}{3}+\dfrac{x^5}{5}$
\end{answer}
\begin{solution}
As we saw in in Example~\eref{CLP101}{eg:SRpsrepD}  of the CLP-2 text,
\[\arctan x = \sum_{n=0}^\infty (-1)^n\frac{x^{2n+1}}{2n+1}\]
which is an alternating series when $x$ is nonzero. If we use its partial sum $S_N$ to approximate $\arctan x$, the absolute error involved is no more than
 \[\frac{|x|^{2(N+1)+1}}{N(n+1)+1} =\frac{|x|^{2N+3}}{2N+3}\]
 We want this error to be at most $10^{-6}$ whenever $-\frac14< x <\frac14$. For this range of $x$ values,
$ \dfrac{|x|^{2N+3}}{2N+3}<\dfrac{1}{(2N+3)4^{2N+3}}$, so we want $N$ that satisfies the  inequality:
\[
\frac{1}{(2N+3)4^{2N+3}}\le \frac{1}{10^5}
\]
A quick check with a calculator shows that $N=2$ suffices.

So, the partial sum
\[ \sum_{n=0}^2 (-1)^n\frac{x^{2n+1}}{2n+1}=x-\frac{x^3}{3}+\frac{x^5}{5}\]
approximates $\arctan x$ to within an error of
$\dfrac{x^7}{7}$.

When $x$ is between $-\frac14$ and $\frac14$,
that error is at most $\dfrac{1}{7\cdot 4^7}=\dfrac{1}{114688}<\dfrac{1}{100000}=10^{-5}$, as desired. (When $x=0$, our approximation is 0, the exact value of $\arctan 0$.)

\end{solution}
